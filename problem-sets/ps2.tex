\documentclass[12pt]{article}

%AMS-TeX packages
\usepackage{amssymb,amsmath,amsthm}
%geometry (sets margin) and other useful packages
\usepackage[margin=1.25in]{geometry}
\usepackage{tikz}
\usepackage{tikz-cd}
\usepackage{graphicx,ctable,booktabs}
\usepackage{mathpartir}
\usepackage{minted}

\usepackage[sort&compress,square,comma,authoryear]{natbib}
\bibliographystyle{plainnat}

%
%Redefining sections as problems
%
\makeatletter
\newtheorem{lemma}{Lemma}
\newenvironment{problem}{\@startsection
       {section}
       {1}
       {-.2em}
       {-3.5ex plus -1ex minus -.2ex}
       {2.3ex plus .2ex}
       {\pagebreak[3]%forces pagebreak when space is small; use \eject for better results
       \large\bf\noindent{Problem }
       }
       }
       {%\vspace{1ex}\begin{center} \rule{0.3\linewidth}{.3pt}\end{center}}
       \begin{center}\large\bf \ldots\ldots\ldots\end{center}}
\makeatother


%
%Fancy-header package to modify header/page numbering
%
\usepackage{fancyhdr}
\pagestyle{fancy}
%\addtolength{\headwidth}{\marginparsep} %these change header-rule width
%\addtolength{\headwidth}{\marginparwidth}
%% \lhead{Problem \thesection}
\chead{}
\rhead{\thepage}
\lfoot{\small\scshape EECS 598: Category Theory}
\cfoot{}
\rfoot{\footnotesize PS 1}
\renewcommand{\headrulewidth}{.3pt}
\renewcommand{\footrulewidth}{.3pt}
\setlength\voffset{-0.25in}
\setlength\textheight{648pt}

%%%%%%%%%%%%%%%%%%%%%%%%%%%%%%%%%%%%%%%%%%%%%%%

%
%Contents of problem set
%

\newcommand{\meet}{\wedge}
\newcommand{\join}{\vee}
\newcommand{\iplmeets}{\textrm{IPL}(\top,\meet)}
\newcommand{\iplneg}{\textrm{IPL}(\top,\meet,\supset)}
\newcommand{\downset}{\mathcal P_{\downarrow}}
\newcommand{\down}{{\downarrow}}

\newcommand{\Set}{\textrm{Set}}
\newcommand{\casePlus}[5]{\textrm{case}_{+}\,{#1}\{\sigma_1{#2}\to {#3}|\sigma_2{#4}\to {#5}\}}
\newcommand{\caseZero}[1]{\textrm{case}_0\,{#1}\{\}}
\newcommand{\id}{\textrm{id}}
\newcommand{\lfpt}{\Sigma_{\textrm{lfpt}}}

\begin{document}

\title{Problem Set 2: Simply Typed Lambda Calculus}
\date{Released: September 11, 2025\\
  %% Updated September 3, 2025\\
  Due: September 25, 2025, 11:59pm
}
\maketitle

Submit your solutions to this homework on Canvas alone or in a group of 2 or
3. Your solutions must be submitted in pdf produced using LaTeX.

\begin{problem}{Laws of Exponentiation}
  We say that $x:A \vdash M : B$ and $y: B \vdash N : A$ form an
  \emph{isomorphism} if $x:A \vdash N[M/y] = x$ and $y:B \vdash M[N/x]
  = y: B$. In this case we say $A$ and $B$ are isomorphic, written $A
  \cong B$.

  Implement the lambda terms for the following isomorphisms. You do
  not need to include the typing derivation, just the terms
  themselves. Provide proofs that (1) and (4) are isomorphisms using
  the equational theory. You do not need to explicitly write the proof
  tree, but just explain what rules are used to justify the
  equalities.
  \begin{enumerate}
  \item $A \Rightarrow B \Rightarrow C \cong (A \times B) \Rightarrow C$
  \item $A \Rightarrow (B \times C) \cong (A \Rightarrow B) \times (A \Rightarrow C)$
  \item $(A \Rightarrow 1) \cong 1$
  \item $(A + B) \Rightarrow C \cong (A \Rightarrow C) \times (B \Rightarrow C)$
  \item $0 \Rightarrow C \cong 1$
  \end{enumerate}
  This gives an idea of why $A \Rightarrow B$ is in category theory
  sometimes called the \emph{exponential} and written $B^A$.
\end{problem}

\begin{problem}{Lawvere's Fixed Point Theorem}
  We saw that Boolean semantics of IPL was incomplete for the
  signature consisting of a single propositional variable, because the
  law of excluded middle is true in all Boolean models but not
  provable.

  In this problem, we consider a similar question for Set-semantics of
  STLC. Define the signature $\lfpt$:
  \begin{enumerate}
  \item Two base type $X$ and $D$
  \item Two function symbols $p : X \to (X \Rightarrow D)$ and $e : (X \Rightarrow D) \to X$
  \item One equation (called ``retraction'') $f : X \Rightarrow D \vdash p(e(f)) = f : X \Rightarrow D$
  \end{enumerate}

  Such a signature may seem strange at first glance, but for any type
  $d$ in a functional programming language with recursive type
  definitions, we can define a corresponding type $X d$ and functions
  $p,E$ that satisfying the retraction axiom. For example in Haskell:
  % TODO: make this look nicer
\begin{minted}{haskell}
data X d = E (X d -> d)

p :: X d -> (X d -> d)
p (E f) = f
\end{minted}

  For this problem, we will prove that this signature only has
  \emph{trivial} models in Set, and show a famous application of this
  triviality. This triviality implies that functional programming
  languages that support unrestricted recursive type definitions
  cannot be given non-trivial Set-theoretic semantics. Later in the
  course we will see that there are non-trivial models in other
  categories besides $\Set$.
  \begin{enumerate}
  \item In STLC generated by $\lfpt$, construct a
    \emph{fixed point combinator} for the type $D$, that is define
    \begin{itemize}
    \item A term $\cdot \vdash \texttt{fix} : (D \Rightarrow D) \Rightarrow D$
    \item That maps any function to a fixed point in that it satisfies the equation:
      \[ f : D \Rightarrow D \vdash f(\texttt{fix}\,f) = \texttt{fix}\,f : D\]
    \end{itemize}
  \item Show that that the equation
    \[ x:D,y:D \vdash x = y : D \]
    is true in any model of $\lfpt$ in $\Set$.
  \item As a corollary of part (2), prove \emph{Cantor's theorem}:
    that there is no surjective function from a set $S$ to its
    powerset $\mathcal P S$. Recall that the powerset of $S$ is
    equivalent to the functions from $S$ to the booleans $2^S$. You
    will need to use the axiom of choice\footnote{A more refined
    version of this argument can prove Cantor's theorem without the
    axiom of choice.}.
  \end{enumerate}
\end{problem}

\begin{problem}{Simultaneous Substitution}
  When all variables are known to be distinct, substitution $M[N/x]$
  can simply be defined as the replacement of $x$ with $N$ everywhere
  in the term $M$. This definition is the STLC version of the
  admissibility of the substitution principle of IPL:

  \[
  \inferrule*[Right=Subst(*)]
  {\Gamma \vdash M : A \and \Gamma, x:A \vdash N : B}
  {\Gamma \vdash N[M/x]}
  \]

  The admissible principle of contraction can also be viewed as a
  textual substitution in the term:

  \[
  \inferrule*[Right=Contraction(*)]
  {\Gamma , x: A, y:A, \Delta \vdash M : C}
  {\Gamma , x: A, \Delta \vdash M[x/y] : C}
  \]

  On the other hand, the use of variables to stand for assumptions
  means that exchange and weakening have no effect on the proof term:
  \begin{mathpar}
  \inferrule*[right=Exchange(*)]
  {\Gamma , y: B, x:A, \Delta \vdash M : C}
  {\Gamma , x:A, y: B, \Delta \vdash M : C}
  \and
  \inferrule*[right=Weakening(*)]
  {\Gamma, \Delta \vdash M : C}
  {\Gamma , x:A, \Delta \vdash M : C}
  \end{mathpar}

  In this exercise you will prove these principles are admissible and
  additionally prove some \emph{equations} about substitution.
  %
  Experience shows that the simplest way to prove these properties
  involves generalizing from ``one-place'' substitutions like $M[N/x]$
  to ``simultaneous'' substitutions that simultaneously substitute for
  \emph{all} free variables in a term.

  We define a (simultaneous) substitution from $\Delta$ to $\Gamma$ to
  be a function $\gamma$ that for each variable $x:A \in \Gamma$
  produces a term $\Delta \vdash \gamma(x) : A$. We write $\gamma :
  \Delta \to \Gamma$ to mean a substitution from $\Delta$ to
  $\Gamma$. Note that this definition is contravariant in that a
  substitution $\gamma : \Delta \to \Gamma$ maps variables in $\Gamma$
  to terms well-typed under $\Delta$.

  We can then define an admissible action of substitution:
  \[
  \inferrule*[Right=GenSubst]
  {\gamma : \Delta \to \Gamma \and
   \Gamma \vdash M : A}
  {\Delta \vdash M[\gamma] : A}
  \]

  Defined by induction on $M$ as follows:
  \begin{align*}
    x[\gamma] &= \gamma(x)\\
    f(M_1,\ldots,)[\gamma] &= f(M_1[\gamma],\ldots)\\
    (M, N)[\gamma] &= (M[\gamma], N[\gamma])\\
    (\pi_j M)[\gamma] &= \pi_j M[\gamma]\\
    ()[\gamma] &= ()\\
    (\sigma_j M)[\gamma] &= \sigma_j M[\gamma]\\
    (\casePlus M {x_1} {N_1} {x_2} {N_2})[\gamma] &= (\casePlus {M[\gamma]} {x_1} {N_1[\gamma, x_1/x_1]} {x_2} {N_2[\gamma, x_2/x_2]})\\
    (\caseZero M)[\gamma] &= \caseZero {M[\gamma]}\\
    (\lambda x. M)[\gamma] &= \lambda x. M[\gamma, x/x]\\
    (M N)[\gamma] &= M[\gamma]\,N[\gamma]\\
  \end{align*}

  Where the notation $\gamma, M/x$ is the extension of the the function to map $x$ to $M$:
  \begin{align*}
    (\gamma,M/x)(y) &= M \tag{if $x = y$}\\
    (\gamma, M/x)(y) &= \gamma(y) \tag{if $x \neq y$}
  \end{align*}

  Define the \emph{identity} substitution $\textrm{id}_\Gamma : \Gamma
  \to \Gamma$ to map each variable in $\Gamma$ to itself:
  $\textrm{id}(x) = x$.

  Given $\gamma : \Delta \to \Gamma$ and $\delta : \Xi \to \Delta$,
  define the \emph{composition} $\gamma \circ \delta : \Xi \to \Gamma$
  as $(\gamma \circ \delta)(x) = (\gamma(x))[\delta]$.

  Below assume $\gamma : \Delta \to \Gamma$, $\delta : \Xi \to
  \Delta$, $\xi : \Xi' \to \Xi$ and $\Gamma \vdash M : A$.

  First, observe (no need to write the proof), that the following
  generalized weakening principle is admissible, if $\Gamma \subseteq
  \Gamma'$, then any term typeable in the smaller context $\Gamma$ is
  also typable in the larger context $\Gamma'$ with the same syntax:
  \begin{mathpar}
    \inferrule*[Right=GenWeak]
               {\Gamma\vdash M : A}
               {\Gamma' \vdash M : A}
  \end{mathpar}
  The proof is by induction on the typing derivation of $M$.

  \begin{enumerate}
  \item Define simultaneous substitutions that correspond to the principles of
    one-place substitution, weakening, exchange and contraction.
  \item Show (by induction on the derivation of $\Gamma \vdash M : A$)
    that $\Delta \vdash M[\gamma] : A$, i.e., that the GenSubst typing
    rule is admissible. You only need to show the following
    representative cases: $M = x$, $M = f(M_0,\ldots)$, $M = \lambda
    x. M'$ and $M = M'\,N$. Where is the GenWeak principle needed?

  \item Show (by induction on $M$) that $M[\id_\Gamma] = M$. Note that this and the
    following equalities are exact syntactic equalities, you will not
    need to use any $\beta\eta$ rules to prove it.
  \item Show (by induction on $M$) that $M[\gamma \circ \delta] = (M[\gamma])[\delta]$
  \item Show that $\gamma \circ \id_{\Delta} = \gamma$ and
    $\id_{\Gamma} \circ \gamma = \gamma$.
  \item Show as a corollary that if $x_2 : A_2 \vdash N_1 : A_1$ and
    $x_3 : A_3 \vdash N_2 : A_2$ and $x_4 : A_4 \vdash N_3 : A_3$ then
    $(N_1[N_2/x_2])[N_3/x_3] = N_1[N_2[N_3/x_3]/x_2]$.
  \item Show that $(\gamma \circ \delta) \circ \xi = \gamma \circ (\delta \circ \xi)$
  \end{enumerate}
\end{problem}

\end{document}
